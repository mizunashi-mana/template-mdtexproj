% refs
\newcommand{\funcref}[1]{式(\ref{func:#1})}
\newcommand{\tblref}[1]{表\ref{table:#1}}
\newcommand{\figref}[1]{図\ref{figure:#1}}
\newcommand{\lstref}[1]{リスト\ref{lst:#1}}

% math symbol
\newcommand{\difd}{\mathrm{d}}
% 案:積分環境
%\newcommand{\intdd}[4]{\int_{#1}^{#2}#3\difd #4}
%\newcommand{\max}{\mathrm{max}}

% caption
\makeatletter
\newcommand{\figcaption}[1]{\def\@captype{figure}\caption{#1}}
\newcommand{\tblcaption}[1]{\def\@captype{table}\caption{#1}}
\makeatother

% caption the number
%\renewcommand{\thesection}{$\circ$}
\renewcommand{\thesubsection}{\arabic{section}.\arabic{subsection}}
\renewcommand{\theequation}{\arabic{section}.\arabic{equation}}
\renewcommand{\thefigure}{\arabic{section}.\arabic{figure}}
\renewcommand{\thetable}{\arabic{section}.\arabic{table}}
\renewcommand{\thefootnote}{\arabic{footnote}}
\makeatletter
\AtBeginDocument{
    \@addtoreset{lstlisting}{section}
    \renewcommand{\lstlistingname}{リスト}
    \renewcommand{\thelstlisting}{\arabic{section}.\arabic{lstlisting}}
}
\@addtoreset{equation}{section}
\@addtoreset{figure}{section}
\@addtoreset{table}{section}
\makeatother

% cite seen
\makeatletter
\DeclareRobustCommand\cite{\unskip
    \@ifnextchar[{\@tempswatrue\@citex}{\@tempswafalse\@citex[]}
}
\renewcommand{\@cite}[2]{$^{[\hbox{\scriptsize{#1\if@tempswa , #2\fi}]}}$}
% \def\@biblabel#1{#1)}
\makeatother

% my columntype
\newcolumntype{C}[1]{>{\centering\arraybackslash}p{#1}}

% lstlisting setting
\lstset{
    frame=trlb,
    breaklines=true,
    showstringspaces=\false, % スペースを可視化するか
    keepspaces=true,
    xleftmargin=3zw,
    xrightmargin=0zw,
    framerule=.2pt,
    columns=[l]{fullflexible},
    numbers=left,
    stepnumber=1,
    numberstyle={\rmfamily},
    numbersep=1em,
    lineskip=-3pt,
    morecomment={[s][{\color[cmyk]{1,0,0,0}}]{/**}{*/}},
    stringstyle={\ttfamily \color[rgb]{0.87,0.07,0.27} }, % 二重引用符内
    commentstyle={\color[rgb]{0.2,0.6,0}\ttfamily\scriptsize}, % コメント
    keywordstyle={\ttfamily \color[rgb]{0,0.5,0.5} }, % 言語のキーワード
    keywordstyle={[2]\color[rgb]{0.2,0.2,0.73}}, % 言語によってあったりなかったりとのこと 色は適当
    keywordstyle={[3]\color[rgb]{0.69,0,0.25}} % 言語によってあったりなかったりとのこと 色は適当
}

% my define
\def\lstinputcode#1#2#3#4{\lstinputlisting[basicstyle={\ttfamily\footnotesize},label={lst:#3},caption={#2},language={#4}]{#1}}